\documentclass{article}
\usepackage{graphicx}
\usepackage{amssymb, amsmath, amsthm}

\author{Elnur Gasanov : 163411}
\title{HW 7 : Support Vector Machine}

\parindent=0mm

\begin{document}
\maketitle

\section{Analysis of Lagrangian}

\subsection{Relation between terms in $\widetilde{L}(\hat{\alpha})$}

Let 

\begin{align*}
\hat{\alpha} & \overset{\text{def}}{=} \arg\max \widetilde{L}(\alpha)\\ 
& = \arg\max \sum\limits_{i=1}^N \alpha_i - \frac12 \sum\limits_{i=1}^N \sum\limits_{j=1}^N \alpha_i \alpha_j y_i y_j \phi^\intercal(x_i) \phi(x_j), \\
& s.t. \ \alpha_i \geq 0, \sum \alpha_i y_i~=~0
\end{align*}

Let us Lagrange multipliers again to solve the problem.

\begin{align*}
\widetilde{\widetilde{L}} (\alpha, \lambda, \nu) = -\sum\limits_{i=1}^N \alpha_i + \frac12 \sum\limits_{i=1}^N \sum\limits_{j=1}^N \alpha_i \alpha_j y_i y_j \phi^\intercal(x_i) \phi(x_j) - \langle \alpha, \lambda \rangle + \nu \left(\sum \alpha_i y_i\right)
\end{align*}

\begin{align*}
\nabla_{\alpha_i} \widetilde{\widetilde{L}}  (\hat{\alpha}, \lambda, \nu)= 0 = y_i \sum\limits_{j=1}^N \hat{\alpha}_j y_j k(x_j, x_i) - 1 - \lambda_i + \nu y_i, \ \forall i \in {1, \dots, n}
\end{align*}

Having multiplied each equation by $\hat{\alpha}_i$ and summing them up, we get

\begin{align*}
\sum\limits_{i=1}^N \sum\limits_{j=1}^N \hat{\alpha}_i \hat{\alpha}_j y_i y_j k(x_j, x_i) - \sum\limits_{i=1}^N \hat{\alpha}_i - \langle \hat{\alpha}, \lambda \rangle + \nu \left(\sum \hat{\alpha}_i y_i\right) = 0
\end{align*}

Since $\sum \hat{\alpha}_i y_i = 0$ and KKT gives $\langle \hat{\alpha}, \lambda \rangle = 0$, we get  

\begin{align}
\boxed{\sum\limits_{i=1}^N \sum\limits_{j=1}^N \hat{\alpha}_i \hat{\alpha}_j y_i y_j k(x_j, x_i) = \sum\limits_{i=1}^N \hat{\alpha}_i}
\end{align}

\subsection{Relation between maximum margin and Lagrange multipliers}

As can be seen from the lecture, $h = \frac{1}{||w||}$. Subsequently

\begin{align*}
\frac1{h^2} = ||w||^2 = ||\sum\limits_{i=1}^N \alpha_i y_i \phi(x_i) ||^2 = \sum\limits_{i=1}^N  \sum\limits_{j=1}^N \alpha_i \alpha_j y_i y_j k(x_i, x_j) =  \sum\limits_{i=1}^N \hat{\alpha}_i = 2 \widetilde{L}(\hat{\alpha}).
\end{align*}

\section{Computational experiments}

\subsection{Default models}

Having followed the task, we have obtained the following results:

\begin{table}[h!]
\centering
\begin{tabular}{|c|c|c|c|}
\hline
 & linear & polynomial & rbf \\
\hline
\# of support vector & 598 & 598 & 598 \\
\hline
AUC-ROC & 0.823 & 0.821 & 0.666 \\
\hline	
\end{tabular}
\caption{Comparison table for default choice of parameters}
\end{table}

\begin{figure*}[h!]
\includegraphics[width=\linewidth]{ROC}
\caption{ROC curves for default values of models with different kernels (linear, polynomial, radial basis functions)}
\end{figure*}

\subsection{5-fold cross validation}

Again having followed the task, the following results were obtained:
\newpage 

\begin{table}[h!]
\centering
\begin{tabular}{|c|c|c|c|}
\hline
 & linear & polynomial & rbf \\
\hline
optimal parameters & {C:0.05} & {C:0.05, $\gamma$:'scale', coef0:0, degree:3} & {C:0.05, $\gamma$:'scale'}\\
\hline
\# of support vector & 604 & 808 & 808 \\
\hline
AUC-ROC & 0.8219 & 0.6159 & 0.6001\\
\hline	
\end{tabular}
\caption{Comparison table for optimal choice of parameters}
\end{table}

\begin{figure*}[h!]
\includegraphics[width=\linewidth]{ROC_kfold}
\caption{ROC curves for optimal according to cross-validation values of models with different kernels (linear, polynomial, radial basis functions) }
\end{figure*}

\subsection{Comparison table}

Let us look at the comparison table

\begin{table}[h!]
\centering
\begin{tabular}{|c|c|c|c|}
\hline
 & linear & polynomial & rbf \\
\hline
default & $\mathbf{0.823}$ & $\mathbf{0.821}$ & 0.666 \\
\hline
k-fold & $\mathbf{0.8219}$ & 0.6159 & 0.6001\\
\hline	
\end{tabular}
\caption{Comparison table of AUC-ROC values in different settings}
\end{table}

In this particular example, default values of models without cross-validation worked well (3 best models: models with linear kernel in both setting and model with polynomial kernel with default values of parameters).

\end{document}